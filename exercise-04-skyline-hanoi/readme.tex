\documentclass[a4paper, 11pt]{article}
\usepackage{comment}
\usepackage{lipsum}
\usepackage{fullpage}
\usepackage[utf8]{inputenc}
\usepackage[T1]{fontenc}
\usepackage{amsmath}
\usepackage{libertine}
\usepackage{libertinust1math}


\begin{document}

\noindent
\large\textbf{Algorithmen und Datenstrukturen} \hfill \textbf{Christoph Stach (555912)} \\
\normalsize Aufgabe 4: Türme von Hanoi \& Skyline Problem \hfill Tom Buhrtz \\
\\
Ich habe diese Aufgaben nur Teilweise implementiert und gelöst da ich die erfordeliche Punktzahl bereits durch Aufgabe 1 und. 2
erreichen sollte. In diesem Dokument ist beschrieben welche Teile dieser Aufgabe gelöst sind.

\section*{Skyline Problem}
1. und 2. Die Datenstrukturen für Häuser und Skyline liegen vor und sind direkt im Quelltext implementiert.
Der Algorithmus ist implementiert jedoch nicht sehr performant.
Häuser können per CMD eingegeben werden und die Skyline (vertikal ;-)) ausgegeben werden.
Eine Beschreibung des Algorithmus in Pseudocode liegt nicht vor, nur in richtigem Java Code


\section*{Türme von Hanoi}
3. und 4. Die Türme von Hanoi sind komplett gelöst und funktinieren einwandfrei. Der Quelltext liegt vor.
Die erforderlichen Datenstrukturen wurden von mir direkt im Quelltext implementiert. \\
5. Bei der Komplexität handelt es sich genau um $O(n) = n^2 - 1$.




\end{document}
