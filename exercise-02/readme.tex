\documentclass[a4paper, 11pt]{article}
\usepackage{comment}
\usepackage{lipsum}
\usepackage{fullpage}
\usepackage[utf8]{inputenc}
\usepackage[T1]{fontenc}
\usepackage{libertine}
\usepackage{libertinust1math}


\begin{document}

\noindent
\large\textbf{Algorithmen und Datenstrukturen} \hfill \textbf{Christoph Stach (555912)} \\
\normalsize Aufgabe 2: Einfach und doppelt verkette Liste \hfill Tom Buhrtz \\



\section*{Einige Methoden obiger einfach verketteter Liste lassen sich (im Gegensatz zum
          Array oder einer doppelt verketteten Liste) effizient (in unterschiedlicher Hinsicht)
          implementieren, andere nicht unbedingt - welche sind das und warum?}

\subsection*{insertFirst()}
Die Methode lässt sich bei der einfach und doppelt verketteten Liste sehr leicht implementieren, da nur ein neuer Kopf
gesetzt werden muss. Im Vergleich zum Array muss hier weniger Aufwand geleistet werden, da ein Array komplett neu
erstellt werden müsste um den neuen Wert zu speichern.

\subsection*{insertLast()}

Ähnlich wie bei \textit{insertFirst()} kann bei der doppelt verketteten Liste einfach ein neuer Fuß gesetzt werden, was
sehr einfach ist, da die doppelt verkette Liste bereits eine Referenz auf den Fuß hat. Bei bei der einfach verketteten
Liste gestaltet sich das komplizierter, weil erst die komplette Liste durchlaufen werden muss um den Fuß zu finden.
Das Array müsste wieder neu erstellt werden.

\subsection*{insert()}

Um Elemente einzufügen müssen beide Listen jeweils bis zum Index durchlaufen werden. Hier können sich die Vorteile einer
doppelt verketten Liste zu nutze gemacht werden, da bei einem Index der größer ist als die Hälfe der Länge der Liste auch
von hinten angefangen werden kann die Liste zu durch laufen. Die implementierung ist dadurch natürlich komplexer aber
man hat unter Umständen im Vergleich zur einfach verketteten Liste Performancevorteile. Das Array müsste wieder neu
erstellt werden.

\subsection*{remove()}

\subsection*{clearAll()}

\subsection*{isEmpty()}

\subsection*{get()}

\subsection*{insertFirst()}

\section*{Analysieren Sie die Komplexität der von ihnen implementierten Sortierverfahren
          allgemein und im speziellen Fall Ihrer Implementierung}

\end{document}
